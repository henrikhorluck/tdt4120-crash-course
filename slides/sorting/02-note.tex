\begin{frame}{Læringsmål}
    \begin{block}{Minner om at følgende gjelder for \emph{alle} algoritmer i pensum}
        \begin{itemize}
            \item Kjenne den formelle definisjonen av det generelle problemet den løser
            \item<2-> Kjenne til eventuelle tilleggskrav den stiller for å være korrekt
            \item<3-> Vite hvordan den oppfører seg; kunne utføre algoritmen, trinn for trinn
            \item<4-> \textbf{!} Forstå korrekthetsbeviset; hvordan og hvorfor virker algoritmen egentlig?
            \item<5-> Kjenne til eventuelle styrker eller svakheter, sammenlignet med andre
            \item<6-> Kjenne kjøretidene under ulike omstendigheter, og forstå utregningen
        \end{itemize}
        \onslide<7->{\textbf{Altså:} Det finnes cheat-sheets, men en må likevel forstå algoritmen til den grad at du kan skrive den i pseudo-kode-form!}
    \end{block}
\end{frame}
