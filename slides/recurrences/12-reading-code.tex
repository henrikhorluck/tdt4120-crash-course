\begin{frame}{Å se kjøretid fra kode}
    \begin{block}{Må øves på}
    Gitt at vi har en funksjon FunctionN, som har kjøretid $\Theta(n)$, hva blir kjøretiden til disse?
    \begin{columns}
\begin{column}{0.5\textwidth}
        \begin{pseudo}*
    \hd{MystiskA}(n) \\
    if $n > 1$ \\+
        \pr{FunctionN}(n) \\
        $a = \pr{Mystisk}(n/2)$ \\
        $b = \pr{Mystisk}(n/2)$ \\
        $c = \pr{Mystisk}(n/2)$ \\
        return $a+b+c$
\end{pseudo}
Vi får rekurrens $T(n) = 3T(n/2) + \Theta(n)$, siden vi kaller funksjonen tre ganger.
\end{column}
\begin{column}{0.5\textwidth}  %%<--- here
    \begin{pseudo}*
    \hd{MystiskB}(n) \\
    if $n > 1$ \\+
        \pr{FunctionN}(n) \\
        return $3 \cdot \pr{Mystisk}(n/2)$
\end{pseudo}
Vi får rekurrense $T(n) = T(n/2) + \Theta(n)$, vi kaller bare funksjonen en gang!
\end{column}
\end{columns}
    \end{block}
\end{frame}
