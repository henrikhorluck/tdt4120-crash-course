\begin{frame}{Algoritmeanalyse}
    \begin{block}{Må øves på}
        Når vi regnet rekurrenser, så antok vi ingenting om hva $n$ var, så vi fant den generelle kjøretiden. Denne kjøretiden kan derimot variere veldig, basert på hvordan input ser ut. 
        \begin{itemize}
            \item Best-case, da antar vi noe om hvordan $n$ ser ut, utover størrelsen på data-en
            \item Worst-case, igjen vi antar noe om $n$
            \item Average-case, da kan vi anta en sannsynlighetsfordeling for input, f.eks. vi antar at vi vanligvis får nesten-sorterte lister. Da må vi vite mer om nøyaktig hva algoritmen brukes til
            \item Amortisert analyse: Da ser vi på gjennomsnitt av kjøretiden over flere kall av algoritmen, men vi antar ingenting om $n$. Eksempel
        \end{itemize}
    \end{block}
\end{frame}
